\section{Promising Algorithms}
Various algorithms can be implemented in a nuclear fuel cycle simulator to 
predict demand in future commodity markets. These can be divided into three 
major branches, Non-optimizing, Deterministic Optimizing, and Stochastic 
Optimizing. These vary in applicability, complexity, compute time, and 
accuracy. Examples of each of the three categories are explained in detail below. 

\subsection{Non-Optimizing(NO)}

Non-optimizing algorithms predict future deployment schedules
based on historical supply-demand data from the simulator. These algorithms do not attempt
to meet demand optimally, thus the name `non-optimizing'. The simple nature of this class of algorithms
allows fast execution time, but only limited precision. The
two non-optimizing algorithms explored in this paper are \gls{ARMA}
and \gls{ARCH} methods. 

\subsubsection{Autoregressive Moving Average (ARMA)}

\gls{ARMA} is a combination of two models, the Autoregressive
and the Moving Average model. The Autoregressive model 
predicts future values with a linear curve fit of the latest
datasets, and the Moving Average method does so by fitting the 
errors \cite{stine_introduction_2011}.

The model is referred to as ARMA(p,q), where the p and q represent 
the order (number of previous time step values fitted) of the autoregressive,
and the moving average parts,
respectively. 
The \gls{ARMA} model can be represented by the following equation:

\begin{align} 
	X_i&=c + \epsilon_t + {\mathlarger{\sum}}_{i=1}^{p}\theta_i \cdot X_{t-i}+
	 {\mathlarger{\sum}}_{j=1}^{q}\phi_j\cdot\epsilon_{t-j}\\
	X_i&= \mbox{next predicted value of the time series}\\
	c&= \mbox{adjustable constant}\\
	\theta_i&=i^{th} \mbox{parameter}\\
	\phi_i&=j^{th} \mbox{parameter of the moving average system}\\
	\epsilon &= \mbox{white noise}
\end{align}


\gls{ARMA} is applied to a `well behaved' set of time series data,
where there is little volatility. This makes \gls{ARMA} a suitable
candidate for demand prediction in the case of slowly changing power demand and 
corresponding reactor deployment.


\subsubsection{Autoregressive Conditional Heteroskedastic (ARCH)}
The \gls{ARCH} model is similar to the \gls{ARMA} model, except that
it uses previous variance terms to calculate current error terms, instead
of the value itself. This allows the model to be used in highly volatile 
time series (e.g. prediction of inflation or stock prices over time 
\cite{bollerslev_generalized_1986}).
The \gls{ARCH} model can be represented by the following equation:

\begin{align}
	\epsilon_t^2 &= \alpha_0+{\mathlarger{\sum}}_{i=1}^{q}\alpha_i\cdot\epsilon_{t-1}^2\\
	\alpha_i&= i^{th} \mbox{parameter of the conditional heteroskedastic model}
\end{align}

A comprehensive fuel cycle simulator must have predictive capabilities which 
can deploy fuel cycle support facilities intelligently even in the face of 
volatile dynamics. Such volatility may arise during transitions between 
technologies, from upsets related to unexpected facility shutdowns, due to the variability 
of an increasingly renewable electric grid, or other nonlinear economic 
drivers.  In these cases, the \gls{ARCH} model is more generically applicable 
and should therefore be preferred over the \gls{ARMA} 
model.



\subsection{Deterministic-Optimizing (DO)}
Deterministic-Optimizing algorithms seek to minimize or maximize an objective 
function with respect to a set of constraints. Compared to stochastic 
optimizing algorithms, deterministic-optimizing algorithms require less 
computing power, and are replicable. The most widely used class of deterministic 
optimizing algorithm is the \gls{LP} model. Simply put, this model optimizes a 
linear objective function where the variables are constrained by multiple 
constraints.

For a demand-driven deployment scenario, an example can be made,
where the following criterion are expressed in mathematical equations.

\begin{itemize}
	\item[] Objective Function : Minimize deployed facility capacity
	\item[] Constraints:
	\begin{enumerate}
		\item Power demand
		\item fuel supply - mining / fabrication / enrichment /reprocessing
	\end{enumerate}
		
\end{itemize}



Two major models that utilize deterministic-optimizing algorithms are
the \gls{GCAM} and the \gls{MARKAL} model. \gls{GCAM} explores consequences to global
change by representing various aspects of economy, energy and the environment 
\cite{edmonds_advanced_1994}.
\gls{MARKAL} is a energy demand driven model that assesses the value of new
energy technologies \cite{fishbone_markal_1981}. The optimization algorithms adopted
by these models are further explained below.

\subsubsection{ \gls{GCAM}}
\gls{GCAM} is a dynamic-recursive model that connects various
social, economical, political decisions to climate change. 
%something here?
It determines the price vector that satisfies all markets by
utilizing the GCAM solver, which finds the root of $\vec{y} = F(\vec{p})$,
where $F(\vec{p})=0$. It does so by using two solver algorithms, the Bisection Method,
and Broyden's Method \cite{patel_gcam_2016-1}. 

\paragraph{Bisection Method}
The Bisection Method finds the root, or $f(x)=0$, by first
getting two x values that have opposite signs. Its flow is as follows:

\begin{enumerate}
	\item Initial upper and lower bounds, $x_l$ and $x_u$, respectively are set,
	so that $f(x_l) \cdot f(x_u) <0 $.
	\item Then the midpoint, $x_m$, is set, by $x_m = \frac{x_l+x_u}{2}$.
	\item Then $f(x_l) \cdot f(x_m)$ is evaluated. If the result is negative,
	then $x_u$ is updated to $x_m$. If the result is positive, $x_l$ is updated
	to $x_m$. If the result is zero, $x_m$ is the root, and the loop is stopped.
	\item Using the new $x_l$ or $x_u$ value, a new $x_m$ value is computed. Then
	$\epsilon_{x_m}$ is calculated to see if it is smaller than a pre-specified tolerance,
	input by the user. If it is, the loop is stopped, and $x_m$ is the output.
	%maybe actually write out $\epsilon_a=\frac{x_m^new-x_m^old}{x_m^new} \cdot 100
	\item If the $\epsilon_{x_m}$ is higher than the tolerance, steps 3 and 4 are repeated.
\end{enumerate}


The Bisection Method has the advantage that it requires little computing time
to 'get close' to a solution. However, in a system of equations with multiple
dimensions, it is sometimes not possible to even have rigorous bounds around 
a solution. Thus, in \gls{GCAM}, the Bisection Method is used to get the 
solver only 'near' the solution, then finds the solution using Broyden's Method.

\paragraph{Broyden's Method}
Broyden's Method is similar to Newton's method, but it saves computing 
time by updating the Jacobian rather than computing it at every iteration. 

\begin{enumerate}
\item Guess initial x value, $x_0$.
\item Evaluate $J(x_0)$ and $f(x_0)$.
\item Calculate $J^{-1}(x_0).$
\item Calculate $x_1 = x_0-J^{-1} \cdot (x_0) \cdot f(x_0)$.
\item Calculate $\Delta F$, $\Delta x$, $\Delta x^T$, $|x|^2$.d
\item Update Jacobian using $J_n = J_{n-1}+ \frac{\Delta F_n - J_{n-1} \cdot \Delta
	 x_n}{|x_n|^2} \cdot \Delta x_n^T $.
\item Repeat from step 4, but with $x_{n+1}$. Repeat so that $f(x_n)$ is less than the 
convergence criterion.
\end{enumerate}

\subsubsection{\gls{MARKAL}}

\gls{MARKAL} uses a general linear-programming algorithm to 
optimize multiple objective functions, namely cost, security and
other environmental functions, given a collection of constraints.
The variables and the functions are listed in detail in the reference
\cite{fishbone_markal_1981} \cite{abilock_users_1979}. 


\paragraph{Multi-objective linear-programming.}
The optimization algorithm in \gls{MARKAL} is a collection
of objective functions, subject to a collection of constraints.
Multi-objective linear programming is used in various applications,
such as economics, finance, engineering design, and power systems
\cite{dorfman_linear_1958}.
It is used in cases where there are competing objectives that need
an optimal decision. For example, a central bank may decide a monetary 
policy to optimize its objectives to lower inflation, unemployment, 
or interest rates. 
 
The mathematical representation is as follows:

\begin{itemize}
	\item[] Minimize:
	\[(\sum_i c_i \cdot X_i) \]
	\item[]subject to:
	\[ \sum_i a_{ji} \cdot X_i \ \leq , \geq \  b_j \] \quad \textrm{and} \quad \[ 	X_i \geq 0 \]
	
\end{itemize}
	
	

Though it looks simple, the objective function and the
constraints are complex models of the relations between
numerous parameters. \gls{MARKAL} creates a matrix
of all the coefficients for the \gls{LP} problem, and is
solved. 


\subsection{Stochastic-Optimizing (SO)}
Stochastic optimization refers to optimization methods
that incorporate randomness in either the objective function
or the constraints \cite{hannah_stochastic_2015}. It aims to find
the roots for the objective function by generating random variables.
This makes it fit for problems that contain uncertainty, which
real world problems mostly are. 
Mathematically, the method attempts to find properties of $f$ without
evaluating $f$ directly, but by using random samples of $F(\theta, \xi).$
The stochastic parameters driving the solution are typically sampled
from probability distributions.

\begin{align}
	f(\theta) &= E_{\xi} [F(\theta, \xi)] \\
	\xi &= \mbox{random variable} \\
	f(\theta) &= \mbox{objective function} \\
	F(\theta,\xi) &= \mbox{ random samples created} \\
\end{align}

The Markov Switching Model and Gaussian Process Regression are key examples of 
the Stochastic-Optimizing category of methods. 


\subsubsection{Markov Switching Model}
The Markov Switching Model depends on the idea that
the future is independent of the past and only dependent
on the present. It utilizes Markov Chains, which 
characterize the probability of a system transitioning
between states.

The Markov Model is used in a wide variety of applications,
such as predicting exchange rates \cite{engel_can_1994}, 
labor markets \cite{krolzig_markov-switching_2002} and search trends \cite{bergamaschi_hidden_2011}.

\begin{align}
	y_t &=a_{0,0} + a_{0,1} S_t + (a_{1,0}+a_{1,1} S_t) y_{t-1} +e_i \\
	P_{ij}&=P(S_t=j | S_{t-1} = i) \\
	y_t &= \mbox{observable random variable} \\
	a &= \mbox{drift term constant} \\
	S_t &= \mbox{hidden state variable} \\ 
	e_i &= \mbox{error} \\
	P_{ij} &= \mbox{probability of a transition in the hidden state variable}  
\end{align}


\subsection{Gaussian Process Regression}
Gaussian Process Regression distributes a function as a Gaussian Process
characterized by a mean function and a covariance function \cite{melo_gaussian_2012}.
The mean value denotes the most probable output, and the covariance is the measure
of confidence. 



%demand for nuclear energy has been random
%production of nuclear fuel will be more or less steady
%the completion date may be random? uncertainty of ~+3 years?
