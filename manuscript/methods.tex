\section{Methods}
At a high level, a meta-review of previous \gls{NFC} gap analyses 
helped to identify the existing simulators and their high 
level capabilities. The strongest comparison of transition scenario code 
capabilities is found in \cite{boucher_international_2010}, 
\cite{brown_identification_2016} and \cite{mccarthy_benchmark_2012}, in which 
international \gls{NFC} simulators to conduct specific transition scenarios 
were tested through systematic benchmarks. In 
\cite{carre_overview_2009} and \cite{hoffman_expanded_2016}, the ability of individual 
simulators to conduct transition scenarios is addressed, however the 
flexibility and performance of their varying algorithms for this capability are 
not addressed.

Primary references for an array of fuel cycle simulators were consulted to 
categorize facility deployment logic present in existing \gls{NFC} simulators. 
Where the details of dynamic demand driven simulation were unknown, this review 
individually investigated available tools.  Similarly, primary references were 
identified which describe analogous capabilities in other domains. That is, 
simulation in economics, operations research industrial processing, and other 
applications were identified and their algorithms for demand-driven deployment 
for commodity production are noted here.

Fuel cycle simulators were categorized as having (1) no automated deployment at all (2) 
deployment based on deterministic forecasting (3) an alternative method. The 
vast majority of existing simulators fall into the first two categories. 
The modeling limitations of both strategies will be discussed. Finally, this 
study will focus on alternative methods in existing simulators and 
promising potential methods which might be implemented in future simulators.  
