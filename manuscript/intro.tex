\section{Introduction}
Nuclear fuel cycle simulation scenarios are most naturally described as 
constrained objective functions. The objectives are often systemic 
demands such as ``achieve 1\% growth for total electricity production 
and reach 10\% uranium utilization''. The constraints take 
the form of nuclear fuel cycle technology availability 
(``reprocessing begins after 2025 and fast reactors first become 
available in 2050''). To match the natural constrained objective form of the 
scenario definition, \gls{NFC} simulators must bring demand responsive 
deployment decisions into the dynamics of the simulation logic.  

In particular, a \gls{NFC} simulator should have the 
capability to deploy supporting fuel cycle facilities to enable 
a demand to be met. Take, for instance, the standard once through fuel 
cycle. Reactors may be deployed to meet a objective power demand. 
However, new mines, mills, and enrichment facilities will also need to be 
deployed to ensure that reactors have sufficient fuel to produce power.  
In many simulators, the unrealistic solution to this problem is to 
simply have infinite capacity support facilities. Alternatively, 
detailing the deployment timeline of all facilities becomes the 
responsibility of the user.

This study has identified flexible, general, and performant algorithms 
available for application to this modeling challenge.  Accordingly, a review 
of current \gls{NFC} simulation tools was conducted to determine their current 
capabilites for demand-driven and transition scenarios.  Additionally, the 
authors investigated promising algorithmic innovations that have been 
successful for similar applications in other domains such as economics and 
industrial engineering Finally, the applicability of such algorithms in the 
context of challenging nuclear fuel cycle simualation questions has been 
described.
