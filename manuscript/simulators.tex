\section{Fuel Cycle Simulators}

\subsection{\gls{VISION}}
\gls{VISION} is a fuel cycle simulation
that is intended to evaluate "what if" scenarios.
It considers the nuclear fuel cycle at a systems level,
not in individual facilities \cite{jacobson_user_2011}.
It tracks materials in different steps of the fuel cycle in an isotopic level,
and can model the life cycle of facilities such as licensing, operation, and
construction. \gls{VISION} provides metrics regarding "waste management,
proliferation resistance, uranium utilization, and economics\cite{jacobson_vision:_2006}."

\subsubsection{System Dynamics Model}
\gls{VISION} uses a system dynamics model, which is a
method to solve dynamic complex systems. The tool used
for \gls{VISION} is PowerSim Studio, a commercial
system dynamics tool \cite{yacout_vision_2006}. The advantage of system dynamics
models are their "clarity of their 'stock-and-flow' structure,
the inclusion of nonlinear and delayed relationships, the
ease of numerical simulation and the potential for highly
interactive operation from 'user-friendly' interfaces." 
The model also utilizes feedback principle, where an interacting
feedback loop forms complex structures and mathematical
equations that define a relationship between variables.

The model, being descriptive in nature, makes it useful for
evaluating option, designing policies, and informing decisions.
