\section{State of The Art}
We reviewed existing public literature regarding 15 fuel cycle simulators to 
determine the extent to which those simulators (1) automatically deploy reactors to 
meet power demand and (2) automatically deploy supporting fuel cycle facilities 
to meet support demands. The fuel cycle simulation tools reviewed included:
\begin{itemize}
        \item CAFCA (MIT) \cite{guerin_impact_2009}
        \item CLASS (CNRS/IRSN) \cite{mouginot_class_2012}
        \item COSI (CEA) \cite{coquelet-pascal_cosi6:_2015,boucher_international_2010}
        \item \Cyclus (UW) \cite{huff_fundamental_2014} % manual
        \item DESAE (Rosatom) \cite{boucher_international_2010}
        \item DANESS (ANL) \cite{van_den_durpel_daness:_2006} % proportional
        \item DYMOND (ANL) \cite{boucher_international_2010}
        \item Evolcode (CIEMAT) \cite{boucher_international_2010}
        \item FAMILY (IAEA) \cite{boucher_international_2010}
        \item MARKAL (BNL) \cite{feng_standardized_2016}
        \item NFCSim (LANL) \cite{schneider_nfcsim:_2005}
        \item NGSAM (ORNL) \cite{aubin_development_2013}
        \item NUWASTE (NWTRB) \cite{garrick_nuclear_2011}
        \item ORION (NNL) \cite{feng_standardized_2016}
        \item VISION (INL) \cite{feng_standardized_2016,boucher_international_2010}
        \item VISTA (IAEA) \cite{iaea_nuclear_2007}
\end{itemize}

We found that automated deployment of supportive fuel cycle facilities is naiive or 
nonexstent in most simulators, including the one under development by the 
authors, \Cyclus.

For the majority of simulators, automated deployment is limited to deploying 
reactors based on changes in power demand. 
For example, as the simulation 
progresses, additional reactors are deployed to meet a power demand projection 
(e.g. 2\% growth over 100 years). 
However, supportive fuel cycle facilities must also be deployed in response to 
(or, more realistically, in preparation for) reactor deployment.  
Typically, current simulators rely on manual deployment of fuel cycle 
facilities. To reduce effort and the likelihood of a failed simulation, the 
user often deploys all potentially necessary fuel cycle facilities at the start 
of the simulation with conservative or infinite throughput capacities. 

Current strategies can be categorized into four genres: 

\begin{itemize}
        \item \textbf{manual:} The user 'guesses' future needed fuel cycle facility 
                deployment needed to support simulated reactors.
                rather than the other way around. 
        \item \textbf{proportional:} Deployment of fuel cycle facilities is in 
                direct proportion with reactor deployments (e.g. for every 10 
                new fast reactors, deploy a new reprocessing plant).
        \item \textbf{constrained reactor deployment:} Deployment of reactors is 
                constrained by the existing and projected feedstock amounts, 
        \item \textbf{predictive:} The simulator projects feedstock needs of 
                current and future deployed reactors based on other heuristics 
                and look-ahead predictions. 
\end{itemize}

Each will be discussed, but the focus of this paper will be to improve on the 
current state of the art implementations of the fourth category, predictive 
methods. 

\subsection{Manual Deployment}
Most software does it this way. Double check each and list them. 

\subsection{Proportional Deployment}

Whether any predictive method can outperform conservative proportional 
deployment is an open question. 

\subsection{Contrained Reactor Deployment}

This may be the most realistic approach (, but is unique to the NFCSim 
model?). Strategies for deployment which are facility agnostic and target 
objective functions related to more wholistic metrics may better inform policy. 
However, deploying reactors based on the existence and projection of supportive 
fuel feedstock is closely reflective of the reality regarding next generation 
reactors. The United States, for example, would not build a first of a kind 
reactor without mature fuel fabrication capabilities. In the case of 
conventional reactors, the realism of this model breaks down, given the broad 
availability of conventional reactor fuel fabrication capacity.  


A constraint not implemented in any existing simulator is related to the impact 
of pricing and economics on this question. If the fuel is available, but only 
for an exorbitant price, the reactor might not be built. 

\subsection{Predictive Deployment}
I think orion does this. 

