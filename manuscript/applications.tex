\section{Other Uses of Dynamic Deployment}
The concept of dynamic demand-driven deployment is not new,
and has been used in various applications. Various algorithms
are used to optimize deployment of facility or utilization
of given sources.

\subsection{Aircraft Dispatch}
To maximize fleet utilization and minimize
operating costs, airlines predict future demands
and optimize their flight schedule and aircraft
type using linear programming methods \cite{berge_demand_1993}. The Demand
Driven Dispatch process evaluates benefits and costs
in various stages of an airline planning stage, like
pricing, aircraft assignment, and crew scheduling \cite{shebalov_practical_2009}.

\subsection{Timber Industry}
A sawmill facility that processes timber contains 
many steps, from sawing, drying to planning. An
agent-based simulation is used to analyse demand-
driven production planning \cite{yáñez_agent-based_2009}.
The facility unit can be represented by a group of many facilities,
for example, source, sawing operations, drying operations, warehouse,
make, and delivery. In the simulation, each agents are given their parameters
and model its procedures.

The demand agent would communicate to the make agent of the demand plan,
which will cause the make agent to communicate to the source agent. Then the
analysed and planned supply chain will be moved back to the make agent and
to the deliver agent.

\begin{figure}
	\includegraphics{width=\linewidth}{timber_process.jpg}
	\caption{Agent Coordination Protocol}
	\label{fig:timber_process}
	\cite{yáñez_agent-based_2009}
\end{figure}
